\documentclass[10pt]{article}

\usepackage{amsmath, amsthm, amssymb, amsfonts}
\usepackage{amsxtra, amscd, geometry, graphicx}
\usepackage{endnotes}
%\usepackage[all,cmtip]{xypic}
\usepackage{mathrsfs}
%\usepackage[pdftex]{hyperref}
%\usepackage[dvips,bookmarks,bookmarksopen,backref,colorlinks,linkcolor={blue},citecolor={blue},urlcolor={blue}](hyperref}

% Makes the margin size a little smaller.
\geometry{letterpaper,margin=1.3in}

% Possible font packages. Choose one and comment out the rest.
\usepackage{times}%
%\usepackage{helvet}%
%\usepackage{palatino}%
%\usepackage{bookman}%

% These are italic.
\theoremstyle{plain}
\newtheorem{thm}{Theorem}
\newtheorem*{thm*}{Theorem}
\newtheorem{prop}{Proposition}
\newtheorem*{prop*}{Proposition}
\newtheorem{conj}{Conjecture}
\newtheorem*{conj*}{Conjecture}
\newtheorem{lem}{Lemma}
  \makeatletter
  \@addtoreset{lem}{thm}
  \makeatother 
\newtheorem*{lem*}{Lemma}
\newtheorem{cor}{Corollary}
  \makeatletter
  \@addtoreset{cor}{thm}
  \makeatother 
\newtheorem*{cor*}{Corollary}

%\newtheorem{lem}[thm]{Lemma}
%\newtheorem{remark}[thm]{Remark}
%\newtheorem{cor}[thm]{Corollary}
%\newtheorem{prop}[thm]{Proposition}
%\newtheorem{conj}[thm]{Conjecture}

% These are normal (i.e. not italic).
\theoremstyle{definition}
\newtheorem*{ack*}{Acknowledgements}
\newtheorem*{app*}{Application}
\newtheorem*{apps*}{Applications}
\newtheorem{defn}{Definition}
\newtheorem*{defn*}{Definition}
\newtheorem{eg}{Example}
  \makeatletter
  \@addtoreset{eg}{thm}
  \makeatother 
\newtheorem*{eg*}{Example}
\newtheorem*{egs*}{Examples}
\newtheorem{ex}{Exercise}
\newtheorem*{ex*}{Exercise}
\newtheorem*{quest*}{Question}
\newtheorem{rem}{Remark}
\newtheorem*{rem*}{Remark}
\newtheorem{rems}{Remarks}
\newtheorem*{rems*}{Remarks}
\newtheorem{prob}{Problem}
\newtheorem*{prob*}{Problem}
\newtheorem*{soln*}{Solution}
\newtheorem{soln}{Solution}


% New Commands: Common Math Symbols
\providecommand{\R}{\mathbb{R}}%
\providecommand{\N}{\mathbb{N}}%
\providecommand{\Z}{{\mathbb{Z}}}%
\providecommand{\sph}{\mathbb{S}}%
\providecommand{\Q}{\mathbb{Q}}%
\providecommand{\C}{{\mathbb{C}}}%
\providecommand{\F}{\mathbb{F}}%
\providecommand{\quat}{\mathbb{H}}%

% New Commands: Operators
\providecommand{\Gal}{\operatorname{Gal}}%
\providecommand{\GL}{\operatorname{GL}}%
\providecommand{\card}{\operatorname{card}}%
\providecommand{\coker}{\operatorname{coker}}%
\providecommand{\id}{\operatorname{id}}%
\providecommand{\im}{\operatorname{im}}%
\providecommand{\diam}{{\rm diam}}%
\providecommand{\aut}{\operatorname{Aut}}%
\providecommand{\inn}{\operatorname{Inn}}%
\providecommand{\out}{{\rm Out}}%
\providecommand{\End}{{\rm End}}%
\providecommand{\rad}{{\rm Rad}}%
\providecommand{\rk}{{\rm rank}}%
\providecommand{\ord}{{\rm ord}}%
\providecommand{\tor}{{\rm Tor}}%
\providecommand{\comp}{{\text{ $\scriptstyle \circ$ }}}%
\providecommand{\cl}[1]{\overline{#1}}%

\renewcommand{\tilde}[1]{\widetilde{#1}}%
\numberwithin{equation}{section}

\renewcommand{\epsilon}{\varepsilon}

\newcommand*\rfrac[2]{{}^{#1}\!/_{#2}}

% This makes the spacing between lines of font a little bigger.  I like the way it looks.
\newcommand{\spacing}[1]{\renewcommand{\baselinestretch}{#1}\large\normalsize}
\spacing{1.2}

\begin{document}

\title{\Large{361B HW2}}
\author{Luke Wukmer}
\date{\today} %to change the date, just put the text you want in place of \today
\maketitle \normalsize \thispagestyle{empty} % remove the page number from the first page

\begin{prob}
Let $\mathbf{x} = ( x_1, x_2, \dots , x_n )^T$ and $f(\bf{x}) = f(x_1, \dots, x_n) : \R^n \rightarrow \R$ denote any differentiable function of $n$ variables.

(a) Given any $\mathbf{y} \in \R^n$ show that
\[
\left(\frac{\mathrm{d}}{\mathrm{d}\epsilon} f(\mathbf{x} + \epsilon \mathbf{y}) \right)\Big|_{\epsilon=0}  =
\big\langle(\nabla f) (\mathbf{x}) ,  \mathbf{y} \big\rangle
\]

 
\end{prob}

\begin{proof}
	Let $f: \R^n \rightarrow \R $ differentiable in $\R^n$. Thus the linear transformation
	$f' ( \mathbf{x} ) \in \mathsf{L} ( \R^n , \R ) $ is completely determined by the partial derivatives w.r.t. $\{x_i\}$ (which are guaranteed to exist); that is:
	
	\[
	f'(\mathbf{x})\mathbf{e}_j = \frac{\partial}{\partial x_j} (\mathbf{x})
	\]
	
	where $\mathbf{e}_j$ is the $j^{th}$ component of the standard basis for $\R^n$.\\
	
	Fix $\bf{x} \in \R^n$. Then for arbitrary $\bf{y} \in \R^n$ , \quad
	$\bf{y} = \displaystyle  \sum^{n} y_j \bf{e}_j $
	\[
	\begin{aligned}
	\Longrightarrow \quad f'(\mathbf{x}) \, \mathbf{y} \, &= \, f'(\mathbf{x})\left(\sum^{n} y_j \mathbf{e}_j\right)\\
								&= \, \sum^{n} y_j f'(\mathbf{x}) \mathbf{e}_j\\
								&= \, \sum^{n} y_j \left(\frac{\partial}{\partial x_j} (\mathbf{x})\right)
								= \langle \mathbf{y} , (\nabla f) (\mathbf{x}) \rangle \\
	\end{aligned}
	\]
	
	Now consider $\displaystyle \left(\frac{\mathrm{d}}{\mathrm{d}\epsilon} f(\mathbf{x} + \epsilon \mathbf{y}\right)$.
	define $\vec{\gamma} : \R \rightarrow \R^n $ by $\vec{\gamma}(\epsilon) = \mathbf{x} + \epsilon \mathbf{y}.$ Then by the chain rule:
	...
	and so
	\[
	\begin{aligned}
	\frac{\mathrm{d}}{\mathrm{d} \epsilon} \left[ f(\mathbf{x} + \epsilon \mathbf{y}) \right] \big|_{\epsilon=0}
	&= \left[ f(\mathbf{x} + \epsilon \mathbf{y}) \right] \big|_{\epsilon=0} \\
	&= f'(x) y \\
	\end{aligned}
	\]
\end{proof}


%Let $(f_n)$ the sequence of functions $f_n:\R\rightarrow\R$ given by
%\[
%	f_n(x) =
%	\begin{cases}
%	0 & \text{if } x \leq 0 \\
%	nx       & \text{if } 0 < x \leq \rfrac{1}{n} \\
%	1		& \text{if } x > \rfrac{1}{n}
%	\end{cases}	
%\]
%
%Then $(f_n)$ converges pointwise on $\R$ to the function $f:\R\rightarrow\R$ by
%\[
%	f(x) = 
%	\begin{cases}
%	0 & \text{if } x \leq 0 \\
%	1 & \text{if } x > 0
%	\end{cases}
%\]
%
%but does not converge uniformly on $\R$.
%
%\end{prob}
%
%\begin{soln*}
%Note that the negation of uniform convergence in $\R$ for a sequence of functions is the statement
%\[
%	(\exists\, \epsilon > 0)(\forall N \in \N)(\exists\, x \in R)(\exists\, n > N)
%		\big[|f_n(x) - f(x)| \geq \epsilon\big]
%\]
%
%Set $\epsilon = \rfrac{1}{2}$ and let $N\in\N$ be given. Then for any $n > N$, consider the point
%$x=\rfrac{\epsilon}{n}$.
%
%We then have
%\[
%	x = \frac{\epsilon}{n} = \frac{1}{2n} < \frac{1}{n} \quad
%	\Longrightarrow \quad f_n(x) = nx = n\left( \frac{1}{2n}\right) = \frac{1}{2}
%\]
%such that
%\[
%	\left| f_n(x) - f(x) \right| = \left|\, \frac{1}{2} - 0 \, \right| = \frac{1}{2} \geq \epsilon = \frac{1}{2}
%\]
%
%Thus for this choice of $\epsilon, N$ we have found a $x$ for any $n>N$ such that the inequality is not less than $\epsilon$; uniform convergence is not possible. \qed
%\end{soln*}


\end{document}
