\documentclass[10pt]{article}

\usepackage{amsmath, amsthm, amssymb, amsfonts}
\usepackage{amsxtra, amscd, geometry, graphicx}
\usepackage{endnotes}
\usepackage{cancel}
%\usepackage[all,cmtip]{xypic}
\usepackage{mathrsfs}
\usepackage{listings}
%\usepackage[pdftex]{hyperref}
%\usepackage[dvips,bookmarks,bookmarksopen,backref,colorlinks,linkcolor={blue},citecolor={blue},urlcolor={blue}](hyperref}

% Makes the margin size a little smaller.
\geometry{letterpaper,margin=1.3in}

% Possible font packages. Choose one and comment out the rest.
\usepackage{times}%
%\usepackage{helvet}%
%\usepackage{palatino}%
%\usepackage{bookman}%

% These are italic.
\theoremstyle{plain}
\newtheorem{thm}{Theorem}
\newtheorem*{thm*}{Theorem}
\newtheorem{prop}{Proposition}
\newtheorem*{prop*}{Proposition}
\newtheorem{conj}{Conjecture}
\newtheorem*{conj*}{Conjecture}
\newtheorem{lem}{Lemma}
  \makeatletter
  \@addtoreset{lem}{thm}
  \makeatother 
\newtheorem*{lem*}{Lemma}
\newtheorem{cor}{Corollary}
  \makeatletter
  \@addtoreset{cor}{thm}
  \makeatother 
\newtheorem*{cor*}{Corollary}

%\newtheorem{lem}[thm]{Lemma}
%\newtheorem{remark}[thm]{Remark}
%\newtheorem{cor}[thm]{Corollary}
%\newtheorem{prop}[thm]{Proposition}
%\newtheorem{conj}[thm]{Conjecture}

% These are normal (i.e. not italic).
\theoremstyle{definition}
\newtheorem*{ack*}{Acknowledgements}
\newtheorem*{app*}{Application}
\newtheorem*{apps*}{Applications}
\newtheorem{defn}{Definition}
\newtheorem*{defn*}{Definition}
\newtheorem{eg}{Example}
  \makeatletter
  \@addtoreset{eg}{thm}
  \makeatother 
\newtheorem*{eg*}{Example}
\newtheorem*{egs*}{Examples}
\newtheorem{ex}{Exercise}
\newtheorem*{ex*}{Exercise}
\newtheorem*{quest*}{Question}
\newtheorem{rem}{Remark}
\newtheorem*{rem*}{Remark}
\newtheorem{rems}{Remarks}
\newtheorem*{rems*}{Remarks}
\newtheorem{prob}{Problem}
\newtheorem*{prob*}{Problem}
\newtheorem*{soln*}{Solution}
\newtheorem{soln}{Solution}


% New Commands: Common Math Symbols
\providecommand{\R}{\mathbb{R}}%
\providecommand{\N}{\mathbb{N}}%
\providecommand{\Z}{{\mathbb{Z}}}%
\providecommand{\sph}{\mathbb{S}}%
\providecommand{\Q}{\mathbb{Q}}%
\providecommand{\C}{{\mathbb{C}}}%
\providecommand{\F}{\mathbb{F}}%
\providecommand{\quat}{\mathbb{H}}%

% New Commands: Operators
\providecommand{\Gal}{\operatorname{Gal}}%
\providecommand{\GL}{\operatorname{GL}}%
\providecommand{\card}{\operatorname{card}}%
\providecommand{\coker}{\operatorname{coker}}%
\providecommand{\id}{\operatorname{id}}%
\providecommand{\im}{\operatorname{im}}%
\providecommand{\diam}{{\rm diam}}%
\providecommand{\aut}{\operatorname{Aut}}%
\providecommand{\inn}{\operatorname{Inn}}%
\providecommand{\out}{{\rm Out}}%
\providecommand{\End}{{\rm End}}%
\providecommand{\rad}{{\rm Rad}}%
\providecommand{\rk}{{\rm rank}}%
\providecommand{\ord}{{\rm ord}}%
\providecommand{\tor}{{\rm Tor}}%
\providecommand{\comp}{{\text{ $\scriptstyle \circ$ }}}%
\providecommand{\cl}[1]{\overline{#1}}%
\providecommand{\tr}{{\sf trace}}%

\renewcommand{\tilde}[1]{\widetilde{#1}}%
\numberwithin{equation}{section}

\renewcommand{\epsilon}{\varepsilon}

\newcommand*\rfrac[2]{{}^{#1}\!/_{#2}}

% This makes the spacing between lines of font a little bigger.  I like the way it looks.
\newcommand{\spacing}[1]{\renewcommand{\baselinestretch}{#1}\large\normalsize}
\spacing{1.2}

% END PREAMBLE %%%%%%%%%%%%%%%%%%%%%%%%%
%%%%%%%%%%%%%%%%%%%%%%%%%%%%%%%%%%%%%%%%


\begin{document}
\lstset{language=Python}

\title{\Large{579 HW3}}
\author{Luke Wukmer}
\date{\today}
\maketitle \normalsize \thispagestyle{empty} % remove the page number from the first page

%%%%% PROBLEM 1
%%%%%%%%%%%%%%%%%%%%%%%%%%%%%%%%%%%%%%%55
\begin{prob}
\end{prob}
\begin{prob*}[\bf{1a}]
\end{prob*}

\begin{proof}
  $f$ convex, so for $ A\vec{w} , A \vec{z} \in \R^{d} $
  \[
    f \left(\theta (A \vec{w}) + (1 - \theta) A \vec{z} \right)
    \; \leq \;
    \theta f( A \vec{w} ) + (1 - \theta) f( A \vec{z} )
  \]
  By linearity of $A$, we can rewrite the left side as
  \[
    f \left(A \left[ \theta \vec{w} + (1 - \theta) \vec{z} \right]\right) 
    \; \leq \;
    \theta f( A \vec{w} ) + (1 - \theta) f( A \vec{z} )
  \]
  Rewriting in terms of $g = f(A\vec(w))$:
  \[
    g \left( \theta \vec{w} + (1 - \theta) \vec{z} \right)
    \; \leq \;
    \theta g(\vec{w}) + (1 - \theta) g(\vec{z})
  \]
  Thus g is convex.
\qedhere  
\end{proof}
%%%% PROBLEM 1B
%%%%%%%%%%%%%%%%%%%%%%%5
\begin{prob*}[\bf{1b}]
\end{prob*}

\begin{proof}
  \[
  \begin{aligned}
    h \left( \theta \vec{x} + (1-\theta) \vec{y} \right)
    &= \sum \alpha_i f_i \left( \theta \vec{x} + (1-\theta) \vec{y} \right) \\
    &\leq \sum \alpha_i \left[ \theta f_i( \vec{x})
                              + (1-\theta) f_i(\vec{y}) \right] \\
    &= \sum \alpha_i \theta f_i( \vec{x})
            + \sum \alpha_i (1-\theta) f_i(\vec{y}) \\
    &= \theta \sum \alpha_i f_i( \vec{x})
            + (1-\theta) \sum \alpha_i  f_i(\vec{y}) \\
    &= \theta h(\vec{x}) + (1-\theta) h(\vec{y}) \\
  \end{aligned}
  \]
  where the inequality follows from the fact that $f_i$ are each convex and 
  $\alpha_i \geq 0$.
\end{proof}
\clearpage
%%%%%% PROBLEM 1C
%%%%%%%%%%%%%%%%%%%%%%%%%%%%
\begin{prob*}[\bf{1c}]
\end{prob*}

\begin{proof}
  \[
  \begin{aligned}
    g(z)   &:= \log(1+e^{z}) \\
    g'(z)  &= \frac{e^{z}}{1+e^{z}} \\
    g''(z) &= \frac{ (1+e^{z}) e^{z} - e^{z} (e^{z}) }{(1+e^{z})^2} \\
           &= \frac{ e^{z} }{ (1+e^{z})^2 } \, \geq \, 0
                \quad \forall z \in \R \\
  \end{aligned}
  \]
\end{proof}

%%%%%%PROBLEM 1D
\begin{prob*}[\bf{1d}]
\end{prob*}

\begin{proof} See attached.
\end{proof}
\end{document}

